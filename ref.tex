\documentclass[russian]{beamer}
\usecolortheme{seahorse} % Выбор цветовой схемы

\usepackage{amsmath} % американское математическое сообщество.
\usepackage{amssymb} % миллион разных значков и готический, ажурный шрифты.
\usepackage{amscd} % диаграммы, графики.
\usepackage{amsthm} % окружения теорем, определений и тд.
\usepackage{physics} % основные физические символы
%\usepackage{latexsym} % треугольники и пьяная стрелка.

%пакеты для шрифтов:
%\usepackage{euscript} % прописной шрифт с завитушками.
%\usepackage{MnSymbol} % Значеки дaxs[i//n, i%n]ельства
%\usepackage{verbatim} % улучшенный шрифт "пишущей машинки".
\usepackage{array} % более удобные таблицы.
\usepackage{multirow} % мультистолбцы в таблицах.
\usepackage{epstopdf} % Векторная графика
\usepackage{longtable} % таблицы на несколько страниц.
%\usepackage{latexsym}
%\usepackage{tikz-feynhand} % Феймановские диограммы
\usepackage{animate}
\usepackage{hyphenat}
%\usepackage{etoolbox}
%\usepackage{collectbox} % Добавляет коробочки, можно складывать туда текст)



%Пакеты для оформления:
%\usepackage[x11names]{xcolor} % 317 новых цветов для текста.
%\usepackage{multicol} % набор текста в несколько колонн.
%\usepackage{breqn} % мультистоки в урвнениях 
%\usepackage{anyfontsize}
\usepackage{graphicx} % расширенные возможности вставки стандартных картинок.
%\usepackage{subcaption} % возможность вставлять картинки в строчку
%\usepackage{caption} % возможность подавить нумерацию у caption.
%\usepackage{wrapfig} % вставка картинок и таблиц, обтекаемых текстом.
\usepackage{cancel} % значки для сокращения дробей, упрощения, стремления.
\usepackage{multirow}
\usepackage{pdfpages} % вставка pdf в документ
%\usepackage{misccorr} % в заголовках появляется точка, но при ссылке на них ее нет.
%\usepackage{indentfirst} % отступ у первой строки раздела
%\usepackage{showkeys} % показывает label формул над их номером.
%\usepackage{fancyhdr} % удобное создание верхних и нижних колонтитулов.
%\usepackage{titlesec} % еще одно создание верхних и нижних колонтитулов

%Пакеты шрифтов, кодировок. НЕ МЕНЯТЬ РАСПОЛОЖЕНИЕ.
\usepackage[utf8x]{inputenc} % кодировка символов.
%\usepackage{mathtext} % позволяет использовать русские буквы в формулах. НЕСОВМЕСТИМО С tempora.
\usepackage[T1, T2A]{fontenc} % кодировка шрифта.
\usepackage[english, russian]{babel} % доступные языки.


%Сокращения
\newcommand{\piv}[2]{\cfrac{\partial #1}{\partial #2}}


%Скобочки
\newcommand{\inrad}[1]{\left( #1 \right)}
\newcommand{\inner}[1]{\left( #1 \right)}
\newcommand{\infig}[1]{\left\{ #1 \right\}}
\newcommand{\insqr}[1]{\left[ #1 \right]}
\newcommand{\ave}[1]{\left\langle #1 \right\rangle}


%% Красивые <= и >=
\renewcommand{\geq}{\geqslant}
\renewcommand{\leq}{\leqslant}

%%Значек выполнятся
\newcommand{\per}{\hookrightarrow}


%% Более привычные греческие буквы
\renewcommand{\phi}{\varphi}
\renewcommand{\epsilon}{\varepsilon}
\newcommand{\eps}{\varepsilon}
\newcommand{\Eps}{\mathfrak{E}}
\newcommand{\com}{\mathbb{C}}
\newcommand{\re}{\mathbb{R}}
\newcommand{\nat}{\mathbb{N}}
\newcommand{\stp}{$\filledmedtriangleleft$}
\newcommand{\enp}{$\filledmedsquare$}


\newcommand{\mergelines}[2]{
\begin{tabular}{llp{.5\textwidth}}
    #1 \\ #2
\end{tabular}
}
\newcommand\tab[1][0.51cm]{\hspace*{#1}}
\newcommand\difh[2]{\frac{\partial #1}{\partial #2}}

\newcommand{\llp}[1]{\lambda_{\Lambda_{#1}}}

    

\title{Оценка редких распадов $B^0_{s} \to \tau^+ \tau^-$}
\author[Карибджанов Матвей]{\underline{Карибджанов Матвей}, Пахлов Павел}

    
\begin{document}
\begin{frame}% первый слайд
    \titlepage
\end{frame}

% Slide 1: Motivation
\begin{frame}{Поиски редкого распада: $B^0_s \to \tau^+ \tau^-$}
  \begin{itemize}
    \item Редкие лептонные распады $B_s$-мезонов --- могут быть чувствительны к новой физике.
    \item Теоретическое предсказание:
    \begin{align*}
      \mathcal{B}(B^0_s \to \tau^+ \tau^-) &\approx 7.7 \times 10^{-7}
    \end{align*}
  \end{itemize}
\end{frame}

% Slide 2: Methodology
\begin{frame}{Реконструкция}
  \begin{itemize}
    \item Данные: 3 fb$^{-1}$ от LHCb (2011--2012).
    \item Канал распада: $\tau^- \to \pi^- \pi^+ \pi^- \nu_\tau$.
    \item Решения:
    \begin{itemize}
      \item Аналитическая реконструкция с использованием топологии распада и масс.
      \item Классификатор на MC сгенерированных по результатам BaBar $\tau^- \to \pi^- \pi^+ \pi^- \nu_\tau$.
    \end{itemize}
  \end{itemize}
\end{frame}

\begin{frame}{Нормировка через $B^0 \to D^- D^+_s$}
  \begin{itemize}
    \item Для получения абсолютного значения используется нормализация на канал:
    \[ B^0 \to D^- D^+_s,\quad D^- \to K^+ \pi^- \pi^-,\quad D^+_s \to K^+ K^- \pi^+ \]
    \item Величина ветвления восстанавливается как:
    \[ \mathcal{B}(B^0_s \to \tau^+ \tau^-) = \alpha_s \cdot s \]
    \item Где нормализационный коэффициент:
    \begin{align*}
    \alpha_s = \frac{\varepsilon_{D_s} \cdot \mathcal{B}_{B^0 \to D_s} \cdot \mathcal{B}_{D^-} \cdot \mathcal{B}_{D^+_s}}
    {N_{DD_s}^{\text{obs}} \cdot \varepsilon_{\tau^+\tau^-} \cdot \mathcal{B}_{\tau}^2} \cdot \frac{f_d}{f_s}
    \end{align*}
    \item Получено: \( \alpha_s = (4.07 \pm 0.70) \times 10^{-5} \)
  \end{itemize}
\end{frame}

\begin{frame}{Результат}
  \begin{itemize}
    \item Верхний предел $B_0$:
    \begin{align*}
      \mathcal{B}(B^0 \to \tau^+ \tau^-) < 2.1 \times 10^{-3}
    \end{align*}
    \item Верхний предел $B_0$:
    \begin{align*}
      \mathcal{B}(B^0_s \to \tau^+ \tau^-)< 6.8 \times 10^{-3}
    \end{align*}
  \end{itemize}
\end{frame}

\begin{frame}{Каналы распада $\tau$}

  \begin{itemize}
    \item $\tau^+ \to e^+ \nu_e \bar \nu_\tau$
    \item $\tau^+ \to \mu^+ \nu_\mu \bar \nu_\tau$
    \item $\tau^+ \to \pi^+ \bar \nu_\tau$
    \item $\tau^+ \to \rho^+ (\pi^+ \pi^0) \bar \nu_\tau$ --- нет
    \item $\tau^+ \to \pi^+ \pi^+ \pi^- \bar \nu_\tau$ --- нет
    \item $\tau^+ \to \rho^+ (\pi^+ \gamma) \bar \nu_\tau$ --- нет
  \end{itemize}
\end{frame}


\begin{frame}{Распады $\tau^+\tau^-$ с двумя нейтрино (сигнальный MC)}
\centering
\begin{tabular}{cc}
\subfigure{\includegraphics[width=0.45\linewidth]{output/E_miss_tau_2nu-1.png}} &
\subfigure{\includegraphics[width=0.45\linewidth]{output/E_miss_tau_2nu-2.png}}

\end{tabular}
{
  \centering
  \subfigure{\includegraphics[width=0.45\linewidth]{output/E_miss_tau_2nu-3.png}}
}
\end{frame}


% ---------- 3 нейтрино ----------
\begin{frame}{Распады $\tau^+\tau^-$ с тремя нейтрино (сигнальный MC)}
  \centering
  \begin{tabular}{cc}
    \subfigure{\includegraphics[width=0.45\linewidth]{output/E_miss_tau_3nu-1.png}} &
    \subfigure{\includegraphics[width=0.45\linewidth]{output/E_miss_tau_3nu-2.png}} \\
    \subfigure{\includegraphics[width=0.45\linewidth]{output/E_miss_tau_3nu-3.png}}&
    \subfigure{\includegraphics[width=0.45\linewidth]{output/E_miss_tau_3nu-4.png}}
  \end{tabular}
\end{frame}


% ---------- 4 нейтрино ----------
\begin{frame}{Распады $\tau^+\tau^-$ с тремя нейтрино (сигнальный MC)}
  \centering
  \begin{tabular}{cc}
    \subfigure{\includegraphics[width=0.45\linewidth]{output/E_miss_tau_4nu-1.png}} &
    \subfigure{\includegraphics[width=0.45\linewidth]{output/E_miss_tau_4nu-2.png}} \\
  \end{tabular}
  {
  \centering
  \subfigure{\includegraphics[width=0.45\linewidth]{output/E_miss_tau_4nu-3.png}}
  }
\end{frame}

% ---------- 2 нейтрино ----------

\begin{frame}{Распады $\tau^+\tau^-$ (GenMC)}
  \begin{itemize}
    \item Тагирующий $B^{tag}_s$ восстановлен --- верно
    \item $mu - 2\sigma < E_{miss} < mu + 2\sigma$
    \item $5.547< M^{tag}_{B_s} < 5.347 GeV$
    \item $N_{chr} = 0$
  \end{itemize}
\end{frame}

\begin{frame}{Распады $\tau^+\tau^-$ с двумя нейтрино (GenMC)}
\centering
\begin{tabular}{cc}
\subfigure{\includegraphics[width=0.45\linewidth]{output/E_gamma_is_2nu-1.png}} &
\subfigure{\includegraphics[width=0.45\linewidth]{output/E_gamma_is_2nu-2.png}}

\end{tabular}
{
  \centering
  \subfigure{\includegraphics[width=0.45\linewidth]{output/E_gamma_is_2nu-3.png}}
}
\end{frame}

% ---------- 3 нейтрино ----------
\begin{frame}{Распады $\tau^+\tau^-$ с тремя нейтрино (GenMC)}
  Тагирующий $B^{tag}_s$ восстановлен --- верно, $mu - 2\sigma < E_{miss} < mu + 2\sigma$, $5.547< M^{tag}_{B_s} < 5.347 GeV$

  \centering
  \begin{tabular}{cc}
    \subfigure{\includegraphics[width=0.45\linewidth]{output/E_gamma_is_2nu-1.png}} &
    \subfigure{\includegraphics[width=0.45\linewidth]{output/E_gamma_is_2nu-2.png}} \\
    \subfigure{\includegraphics[width=0.45\linewidth]{output/E_gamma_is_2nu-3.png}}&
    \subfigure{\includegraphics[width=0.45\linewidth]{output/E_gamma_is_2nu-4.png}}
  \end{tabular}
\end{frame}


% ---------- 4 нейтрино ----------
\begin{frame}{Распады $\tau^+\tau^-$ с четырьмя нейтрино (GenMC)}
  Тагирующий $B^{tag}_s$ восстановлен --- верно, $mu - 2\sigma < E_{miss} < mu + 2\sigma$, $5.547< M^{tag}_{B_s} < 5.347 GeV$

  \centering
  \begin{tabular}{cc}
    \subfigure{\includegraphics[width=0.45\linewidth]{output/E_gamma_is_4nu-1.png}} &
    \subfigure{\includegraphics[width=0.45\linewidth]{output/E_gamma_is_4nu-2.png}} \\
  \end{tabular}
  {
  \centering
  \subfigure{\includegraphics[width=0.45\linewidth]{output/E_gamma_is_4nu-3.png}}
  }
\end{frame}

% ---------- 2 нейтрино ----------

\begin{frame}{Распады $\tau^+\tau^-$ (GenMC)}
  \begin{itemize}
    \item Тагирующий $B^{tag}_s$ восстановлен --- не
    \item $mu - 2\sigma < E_{miss} < mu + 2\sigma$
    \item $5.547< M^{tag}_{B_s} < 5.347 GeV$
    \item $N_{chr} = 0$
  \end{itemize}
\end{frame}

\begin{frame}{Распады $\tau^+\tau^-$ с двумя нейтрино (GenMC)}
Тагирующий $B^{tag}_s$ восстановлен --- неверно, $mu - 2\sigma < E_{miss} < mu + 2\sigma$, $5.547< M^{tag}_{B_s} < 5.347 GeV$

\centering
\begin{tabular}{cc}
\subfigure{\includegraphics[width=0.45\linewidth]{output/E_gamma_not_2nu-1.png}} &
\subfigure{\includegraphics[width=0.45\linewidth]{output/E_gamma_not_2nu-2.png}}

\end{tabular}
{
  \centering
  \subfigure{\includegraphics[width=0.45\linewidth]{output/E_gamma_not_2nu-3.png}}
}
\end{frame}

% ---------- 3 нейтрино ----------
\begin{frame}{Распады $\tau^+\tau^-$ с тремя нейтрино (GenMC)}
  Тагирующий $B^{tag}_s$ восстановлен --- неверно, $mu - 2\sigma < E_{miss} < mu + 2\sigma$, $5.547< M^{tag}_{B_s} < 5.347 GeV$

  \centering
  \begin{tabular}{cc}
    \subfigure{\includegraphics[width=0.45\linewidth]{output/E_gamma_not_2nu-1.png}} &
    \subfigure{\includegraphics[width=0.45\linewidth]{output/E_gamma_not_2nu-2.png}} \\
    \subfigure{\includegraphics[width=0.45\linewidth]{output/E_gamma_not_2nu-3.png}}&
    \subfigure{\includegraphics[width=0.45\linewidth]{output/E_gamma_not_2nu-4.png}}
  \end{tabular}
\end{frame}


% ---------- 4 нейтрино ----------
\begin{frame}{Распады $\tau^+\tau^-$ с тремя нейтрино (сигнальный MC)}
  Тагирующий $B^{tag}_s$ восстановлен --- неверно, $mu - 2\sigma < E_{miss} < mu + 2\sigma$, $5.547< M^{tag}_{B_s} < 5.347 GeV$
  \centering
  \begin{tabular}{cc}
    \subfigure{\includegraphics[width=0.45\linewidth]{output/E_gamma_not_4nu-1.png}} &
    \subfigure{\includegraphics[width=0.45\linewidth]{output/E_gamma_not_4nu-2.png}} \\
  \end{tabular}
  {
  \centering
  \subfigure{\includegraphics[width=0.45\linewidth]{output/E_gamma_not_4nu-3.png}}
  }
\end{frame}

\end{document}